\documentclass[a4paper]{exam}
\usepackage[utf8]{inputenc}\usepackage{amsthm}\usepackage{bm}
\usepackage{enumitem}
\usepackage{amsmath}\usepackage{amssymb}
\usepackage{tikz}
%\printanswers %Comment this line to print only questions
\author{}
\pdfinfo{
   /Author (Arunima Prasad JS)
   /CreationDate (D:2022010100000)
   /Keywords (CS, GECBH, ECE)
}
\usepackage[pdftex,
            pdftitle={Control System Questions},
            pdfsubject={Control System},
            pdfproducer={Not a Robot},
            pdfcreator={GEC Barton Hill, TVM}]{hyperref}
\usepackage{hyperxmp}
\usepackage{graphicx}
\title{\underline{\textbf{Control System Questions}}\\Multiple Choice Questions (MCQ)  \\  }
\begin{document}

\maketitle

\begin{questions}
\question
The term ‘reset control’ refers to:
\begin{enumerate}[label=(\Alph*)]
\item Integral control
\item Derivative control
\item Proportional control
\item None of the above
\end{enumerate}
\begin{solution}
(A)
\end{solution}

\question
The transfer function $\displaystyle\frac{1+0.5s}{1+s}$ represent a:
\begin{enumerate}[label=(\Alph*)]
\item Lag network
\item Lead network
\item Lag–lead network
\item Proportional controller
\end{enumerate}
\begin{solution}
(A)
\end{solution}

\question
While designing controller, the advantage of pole–zero cancellation is:
\begin{enumerate}[label=(\Alph*)]
\item The system order is increased
\item The system order is reduced
\item The cost of controller becomes low
\item System’s error reduced to optimum levels
\end{enumerate}
\begin{solution}
(B)
\end{solution}

\question
A proportional controller leads to:
\begin{enumerate}[label=(\Alph*)]
\item infinite error for step input for type 1 system
\item finite error for step input for type 1 system
\item zero steady state error for step input for type 1 system
\item zero steady state error for step input for type 0 system
\end{enumerate}
\begin{solution}
(C)
\end{solution}

\question
The state-space representation of a system is given
by:\\
$
\boldsymbol{\dot{x}}=
\begin{bmatrix}
-1 & 0\\
0 & -2
\end{bmatrix} 
x(t)+
\begin{bmatrix}
1 \\
1 
\end{bmatrix}
u(t), y(t) = 
\begin{bmatrix}
1 & 1\\
\end{bmatrix} 
x(t)
\ \ \ 
$
\\The transfer function of this system is
\begin{enumerate}[label=(\Alph*)]
\item ${(s^2+3s+2)}^{-1}$
\item ${(s+2)}^{-1}$
\item $s{(s^2+3s+2)}^{-1}$
\item ${(s+1)}^{-1}$
\end{enumerate}
\begin{solution}
(D)
\\ $T(s)=
\begin{bmatrix}
1 \\
1 
\end{bmatrix}
{(s\boldsymbol{I}-\boldsymbol{A})^{-1}}
\begin{bmatrix}
1 \\
0 
\end{bmatrix}\\
(s\boldsymbol{I}-\boldsymbol{A})^{-1}=
\begin{bmatrix}
\frac{1}{s+1} & 0\\
0 & \frac{1}{s+2}
\end{bmatrix} 
\\T(s)=
\begin{bmatrix}
1 \\
1 
\end{bmatrix}
\begin{bmatrix}
\frac{1}{s+1} & 0\\
0 & \frac{1}{s+2}
\end{bmatrix}
\begin{bmatrix}
1 \\
0 
\end{bmatrix}
=\frac{1}{s+1}
$
\end{solution}


\question
A Lag network for compensation normally consists of:
\begin{enumerate}[label=(\Alph*)]
\item R, L and C elements
\item R and L elements
\item R and C elements
\item R only
\end{enumerate}
\begin{solution}
(C)
\end{solution}

\question
The phase margin of a system with the open loop transfer function:\\
$\displaystyle G(s)H(s)=\frac{(1-s)}{(1+s)(3+s)}$ is
\begin{enumerate}[label=(\Alph*)]
\item $68.3^{\circ}$
\item $90^{\circ}$
\item $0^{\circ}$
\item $\infty$
\end{enumerate}
\begin{solution}
(D)\\
$|{GH(j\omega)|\neq 1}$, for any value of $\omega$. Thus phase margin is $\infty$
\end{solution}

\question
The correct sequence of steps needed to improve system stability is:
\begin{enumerate}[label=(\Alph*)]
\item reduce gain, use negative feedback, insert derivative action
\item reduce gain, insert derivative action, use negative feedback
\item insert derivative action, use negative feedback, reduce gain
\item use negative feedback, reduce gain, insert derivative action
\end{enumerate}
\begin{solution}
(D)
\end{solution}

\question
A lead compensating network
\begin{enumerate}[label=(\alph*)]
\item improves response time
\item stabilizes the system with low phase margin
\item enables moderate increase in gain without affecting stability
\item increases resonant frequency
\end{enumerate}
In the above statements, correct are:
\begin{enumerate}[label=(\Alph*)]
\item (a) and (b)
\item (a) and (c)
\item (a), (c) and (d)
\item (a), (b), (c) \& (d)
\end{enumerate}
\begin{solution}
(D)
\end{solution}

\question
The pole–zero plot given in fig. is that of a:\\



\tikzset{every picture/.style={line width=0.75pt}} %set default line width to 0.75pt        

\begin{tikzpicture}[x=0.75pt,y=0.75pt,yscale=-1,xscale=1]
%uncomment if require: \path (0,310); %set diagram left start at 0, and has height of 310

%Straight Lines [id:da21843549461658163] 
\draw    (365.5,213.17) -- (614.56,214.68) ;
%Straight Lines [id:da07863626362257592] 
\draw    (539.64,107.04) -- (538.12,273.81) ;
%Shape: Ellipse [id:dp339933904313815] 
\draw  [fill={rgb, 255:red, 255; green, 255; blue, 255 }  ,fill opacity=1 ] (427.84,213.47) .. controls (427.84,209.83) and (430.9,206.88) .. (434.69,206.88) .. controls (438.48,206.88) and (441.54,209.83) .. (441.54,213.47) .. controls (441.54,217.1) and (438.48,220.05) .. (434.69,220.05) .. controls (430.9,220.05) and (427.84,217.1) .. (427.84,213.47) -- cycle ;
%Shape: Ellipse [id:dp7316271746989322] 
\draw  [fill={rgb, 255:red, 255; green, 255; blue, 255 }  ,fill opacity=1 ] (451.83,213.47) .. controls (451.83,209.83) and (454.9,206.88) .. (458.68,206.88) .. controls (462.47,206.88) and (465.54,209.83) .. (465.54,213.47) .. controls (465.54,217.1) and (462.47,220.05) .. (458.68,220.05) .. controls (454.9,220.05) and (451.83,217.1) .. (451.83,213.47) -- cycle ;
%Straight Lines [id:da42256600779120435] 
\draw    (403.27,210.19) -- (414.13,217.37) ;
%Straight Lines [id:da39992866607843136] 
\draw    (404.48,218.06) -- (412.32,208.52) ;
%Straight Lines [id:da9151605699430307] 
\draw    (476.96,210.53) -- (487.81,217.92) ;
%Straight Lines [id:da600067973776121] 
\draw    (478.17,218.61) -- (486.01,209.07) ;

% Text Node
\draw (508.16,104.82) node [anchor=north west][inner sep=0.75pt]    {$\mathbf{j\omega }$};
% Text Node
\draw (596.92,217.83) node [anchor=north west][inner sep=0.75pt]    {$\mathbf{\sigma }$};
\end{tikzpicture}

\begin{enumerate}[label=(\Alph*)]
\item PID controller
\item PD controller
\item Integrator
\item Lag–lead compensating network
\end{enumerate}
\begin{solution}
(D)
\end{solution}


\end{questions}
\end{document}
