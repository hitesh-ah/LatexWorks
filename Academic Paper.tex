\documentclass[10pt,twocolumn, a4paper]{article}
% \documentclass[10pt, a4paper]{article}
\usepackage{amsmath, authblk}
\usepackage{graphicx}
\usepackage[pdftex,
            pdftitle={Academic Work},
            pdfsubject={Electronics and Communication},
            pdfproducer={Not a Robot},
            pdfcreator={GEC Barton Hill, TVM}]{hyperref}
\usepackage[round]{natbib}
\begin{document}
\title{TITLE}
% \author{Hitesh, Author2}
% \date{Dept\\Institution, City with Postal Code, Country}
\author[1]{Aman}
\author[2]{Anish}
\author[1]{Arul}
\affil[1]{Dept, Insti1, Addr1}
\affil[2]{Dept2, Insti2, Addr2}
% \date{}
%ABSTRACT AS SINGLE COLUMN%
\twocolumn[
\begin{@twocolumnfalse}
\maketitle
\begin{abstract}
    The World Before the Flood is associate degree oil-on-canvas painting by English creator William Etty, 1st exhibited in 1828. It depicts a scene from John Milton' Paradise Lost within which Adam sees a vision of the planet in real time before the nice Flood. The painting illustrates the stages of courting as represented by Milton: a gaggle of men choose wives from a group of performing arts women, take their chosen lady from the group, and quiet down to married life. Behind them looms an oncoming storm, a logo of the destruction which the dancers and lovers are on the point of bring upon themselves.
    \noindent \textbf{Subject Classification:} Mathematics 
\end{abstract}
\end{@twocolumnfalse}
]
\section{Introduction}\label{sec1}
\textbf{N-Treat Technology:}
To prevent sludge and sewage from 25 storm water drains between Bandra and Dahisar from flowing into the sea, Brihanmumbai Municipal Corporation (BMC) has planned in-situ treatment of sewage from the drains through N-Treat technology.
\subsection{Facts}\label{sec1.1}
The project will be undertaken with the help of Indian Institute of Technology-Bombay’s (IIT-B) N-Treat Technology.
\begin{equation}\label{eq1}
    0+3*4=12
\end{equation}
\section{Mathematical Equations}\label{sec2}
$\Rightarrow$
\begin{equation}\label{eq2}
    \begin{aligned}
      \mu_{f} &=\frac{1}{f^2}\int_\infty^\infty t f[t]^2dt\\
      \sin^{2}x+\cos^{2}x &=1
    \end{aligned}
\end{equation}
The above Equation~\eqref{eq2} can be used in our advantage in triangles. 
\begin{align}
    e&=\lim_{z\to \infty} \left(1+\frac{1}{z} \right)^z \quad\mbox{and}\label{eq5}\\1&=\cosh{y}
\end{align}
Using the expression from~
\eqref{eq2}, can rewritten as:
\begin{equation}\label{4}
    \ln\left[\lim_{z\to\infty}\left(1+\frac{1}{z}\right)^z\right]
\end{equation}
From the previous section (~\ref{sec1})
\\
\begin{table}[!h]
\centering
\begin{tabular}{lcrr@{.}l}
\hline
Item & No. & Rate & Rs & P\\
\hline
Shirt & 3 & 300.25 & 900 & 75 \\
Pants & 4 & 1456.25 & 10900 & 75 \\
\hline
\end{tabular}
    \caption{Caption}
    \label{tab:my_label}
\end{table}
\section{Conclusion}\label{concl}
Other works has shown substantial increment in machine learning output. The future is filled with new horizon of successive steps.\citep*{Graham1995} or \citep*{Thomas2008}.
% \begin{thebibliography}{00}
% \bibitem{nothing}
% \textit{www.wikipedia.org}
% \end{thebibliography}
\newpage
\bibliographystyle{plainnat}
\setcitestyle{numbers}
\bibliography{BibTeX_files}
\end{document}